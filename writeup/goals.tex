\section{goals and timeline}
My main interest is in reproducing the panorama system implemented by the original paper, in which the authors reported an implementation of 6,000 lines of GO code. I envision this part alone nontrivial amount of work. As a first step, \textbf{I would aim at delivering the key insight of this work and making it functionable to collect and analyze errors in a distributed system, with a projected amount of 1k to 2k lines of GO code}. \textbf{I would also reuse the author's implementation of the basic types and focus on reimplementing the core functions only}. I would detail what is imported from the original project and what is written by myself in the final report. 
Here I divide my goals into two categories: \textit{Must-have} and \textit{Might-have}. I would make sure to deliver the items in \textit{Must-have} and might deliver those in \textit{Might-have} depending on the time available. Below I detail my goals.
\subsection{Panorama System}
The panorama project consists of two parts: the panorama system and a code injection tool that automatically inserts code based on offline static analysis. The former is implemented in GO and the latter is implemented on top of Soot and AspectJ. As I previously noted, reproducing panorama system itself is nontrivial, and my interest is in distributed systems not static analysis. Thus, I do not consider recreating the code injection tool high priority. If I end up with some free time, I might consider extending panorama with more functionalities. Below, I summarize my goals as in Table ~\ref{table:1} (assuming we have 4 weeks of time).
% \paragraph{Must-have} 
% \begin{itemize}
% \item Key features of Panorama System (ETA 3 $\sim$ 3.5 weeks)
% \end{itemize}
% \paragraph{Could-have}
% \begin{itemize}
%   \item Code Injection Tool (ETA unknown)
%   \item Extend Panorama with more functionalities (ETA unknown)
% \end{itemize}
\begin{table}[h!]
  \centering
  \begin{tabular}{ |c|c|c|c|c| } 
  \hline
  Goals & Type & Estimated LOC & Timeline \\
  \hline
  Key part of Panorama & \textit{\textbf{Must-have}} & 1k - 2k & Week2/3\\ 
  The rest of Panorama & \textit{Might-have} & 2k - 3k & Week4\\ 
  \hline
  \end{tabular}
  \caption{Goals with regard to implementing the Panorama System assuming we have 4 weeks of time.}
  \label{table:1}
\end{table}
\subsection{Test Environment and Cases}
The authors setup the experiments in a real-world setting: a cluster of 20 physical machine, each with a 2.4 GHz CPU, 64 GB of RAM, and a 480 GB SATA SSD. They then evaluated Panorama with four widely-used distributed systems: ZooKeeper, Hadoop, HBase, and Cassandra. Clearly, these are settings I would not able to achieve. To make my project tractable, \textbf{I would not reproduce any bugs in a real-world workflow, instead I would introduce them in an artificial and simple way}, as detailed in section ~\ref{ssec:rob}. Below I detail my goals and timeline for setting up test environment and cases. 
% \paragraph{Must-have}
% \begin{itemize}
% \item Evaluate on simple cases (ETA 3 $\sim$ 4 weeks)
% \end{itemize}
% \paragraph{Could-have}
% \begin{itemize}
%   \item Reproduce the evaluation environments used by the authors (ETA unknown)
%   \item Reproduce the evaluation cases in the paper (ETA unknown)
%   \item Extend the evaluation cases used by the authors (ETA unknown)
% \end{itemize}

\begin{table}[h!]
  \centering
  \begin{tabular}{ |c|c|c|c|c| } 
  \hline
  Goals & Type & Estimated LOC & Timeline \\
  \hline
  Reproduce one bug & \textit{\textbf{Must-have}} & N/A & \textbf{Done}\\ 
  Run Panorama & \textit{\textbf{Must-have}} & N/A & \textbf{Done}\\ 
  Find the bug in Panorama & \textit{\textbf{Must-have}} & N/A & \textbf{Done} \\ 
  Test my system with the bug& \textit{\textbf{Must-have}} & N/A & Week4 \\ 
  Add more bugs& \textit{Might-have} & N/A & Week4 \\ 
  \hline
  \end{tabular}
  \caption{Goals with regard to setting up the test environment and cases assuming we have 4 weeks of time.}
  \label{table:2}
\end{table}



