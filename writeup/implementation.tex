\section{Implementation}
We build LittlePanorama on top of classes and types defined in Panorama. We keep the same program interfaces when possible. Panorama defines and implements most classes, types and interfaces using protocol buffers ~\cite{Protocol27:online}. Reusing its code largely reduces our workload so that we do not have to implement everything from scratch. Retaining the same set of interfaces allows easy alternation between Panorama and LittlePanorama. 

We implement LittlePanorama server in $\sim$1, 200 lines of Go code. This number excludes code imported from Panorama, modification made to ZooKeeper's Java source code and other scripts written in bash. Additionally, as we previous mentioned, failures used to evaluated Panorama are not disclosed. In contrast, we publish the two gray failures we acquired from the authors. Details regarding these two failures are in Section ~\ref{sec:reproducebugs}. Further, the authors evaluate these failures in real-world settings, making it hard to reproduce their results. On the contrary, as we detail later in Section ~\ref{sec:reproducebugs}, we modify ZooKeeper's source code and inject these failures in a simple way such that it can be easily reproduced. To encourage further scientific exploration, we are open-sourcing our implementation of LittlePanorama as well as our customized version of ZooKeeper \footnote{https://github.com/alexliu0809/LittlePanorama}.

Since we reuse code from Panorama, it is also important to distinguish code written by us and code imported. Table ~\ref{tab:implementation} lists modules that are included in LittlePanorama and whether they are imported from Panorama.
\begin{table}[!tb]
\begin{tabular}{p{0.2\columnwidth}p{0.52\columnwidth}p{0.18\columnwidth}}%{l|l|l}

\toprule
    \textbf{Module} & \textbf{Description} & \textbf{Imported} \\
\midrule
      build & generated by protocol buffers & Yes \\
      cmd & main function & No \\
      decision & decision engine & No \\
      exchange & propagation protocol  & No \\
      service & type/class defs & Yes \\
      store & data storage & No \\
      type & additional types & No \\
      util & utility functions & No \\
      vendor & other third-party modules & Yes \\
\bottomrule
\vspace{0.5em}
\end{tabular}
\caption{Modules included in LittlePanorama and whether they are imported from Panorama.}
\label{tab:implementation}
\end{table}