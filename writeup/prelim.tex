\section{Preliminary Progress}
We have made some initial progress setting up the testing environment and reproducing one test case to ensure that our end goal is achievable. First, we contacted Ryan, who is the lead author of panorama, and received practical advice from them. Following Ryan's suggestions, we succeeded in reproducing one important but easy to set up bug in the Zookeeper \cite{httpsiss99:online}. We now are fairly confident that we would be able to have real cases to evaluate. We detail our progress below.
\subsection{Acquiring Test Cases Used in the Paper}
On Thursday, I sent an email to Ryan Huang at JHU asking for test cases he used in the paper. Fortunately, Ryan was kind enough to respond to my email and attach a few simple test cases he used in the paper. ZOOKEEPER-2201 is one of them \cite{httpsiss99:online}. It is a subtle gray failure bug found in Zookeeper version 3.4.6 and 3.5.0. Ryan indicated that this bug might be one of the easiest bugs to reproduce. Thus, we worked on this one first.
\subsection{Setting Up Zookeeper Ensemble}
We set up a three-server Zookeeper ensemble on the EC2 VM provided to us. First, we downloaded the source code of Zookeeper version 3.4.6 and made a customized version of it. Next, we deployed three Zookeeper servers listening on different ports, forming a Zookeeper ensemble.
\subsection{Reproducing ZOOKEEPER-2201}
\label{subsec:rbug}
Reproducing ZOOKEEPER-2201 in a real-world setting is nontrivial. We have two issues there. First, we would face great difficulty simulating a real-world scenario, which requires a 20-node cluster plus GBs of data. Second, Ryan noted in his correspondence with us that faults typically need to be injected at careful timing in a real environment. Neither of these two issues could be resolved in a tractable amount of time. 

That being said, there is one thing we could do here. Our insight here is that we do not have to be in a real-world situation to trigger ZOOKEEPER-2201. ZOOKEEPER-2201 is reported and fixed in 2015. Along with fixing the bug, the maintainers also attached a test case that examined this specific bug. In that test case, they intentionally introduced a buggy function that would trigger ZOOKEEPER-2201 if it were not fixed. This is convenient for us: we can just use this function to trigger ZOOKEEPER-2201. As we previously noted, we modified the Zookeeper source code and injected this function. We could control when this function is called, and thus controlling when this bug is triggered. 
\subsection{Future Plan on Configuring Test Cases}
As I am now working alone on this project, I am envisioning that I would only test my code on one or two bugs in Zookeeper, which I have already setup on my EC2 VM.

However, being able to trigger a bug is not good enough. We want to make sure that Panorama could detect it and use Panorama as an oracle to test against my implementation of Panorama. To achieve this, two issues need to be addressed. On the one hand, obviously, we need to get Panorama server running. On the other hand, we need to inject code into Zookeeper server so that it could report anything it observes to Panorama server. These two issues are the prerequisite for Panorama to be able to detect any bug. I plan to work on these two issues in the upcoming week.