\section{Background And Motivation}
Modern distributed systems usually consist of numerous components and have high complexity. As a result, properly and promptly capturing failures in production environments is perceived as an annoying task ~\cite{dean2009designs, liu2007wids}. While a rich literature has researched the problem of failure detection, it remains challenging. A recent paper at OSDI '18 ~\cite{huang2018capturing}, titled ``Capturing and Enhancing In Situ System Observability'' by Huang et al., sought to provide a solution to this problem. The authors designed Panorama, a system for detecting production failures in distributed systems. They leveraged the key insight that \textbf{system observability could be improved by making each component an observer of other components with which it interacts}. Each observer would provide observations on the components it interacts with. Taking advantage of these first-hand observations, the authors came up with a simple detection algorithm that achieved high detection accuracy. Their evaluation on 15 real-world failures suggested that Panorama were able to detect and localize failures in a fast manner, whereas existing detectors would only detect a fraction of the failures and run much longer. A follow-up work by the same group of researchers received a best paper award at NSDI '20, demonstrating the power of this idea \cite{246326}.

