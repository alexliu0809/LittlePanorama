\section{Discussion}
The Panorama system essentially consists of two parts: Pano-rama servers and observers. We reimplement the core functions of Panorama servers and do not encounter any significant obstacles in that process. The design and program logic of Panorama seems straightforward, yet very powerful. However, the process of injecting API hooks into ZooKeeper is less pleasant. We are not sure why the authors do not publish the tool that injects API hooks automatically. We realize that this tool is actually of great importance, as where the hooks are injected has a big impact on the detection speed. Injecting hooks at critical places could potentially boost bug detection, while injecting recklessly would result in nothing but overhead. Sadly, the authors do not elaborate too much on how they insert hooks, besides a few short sections. Another thing that the authors could have done is to disclose the bugs that they use in their evaluation, even though readers might not be able to reproduce them given their complexity. 

\section{Conclusion}
In summary, in this work, we present LittlePanorama, a system that implements core modules of Panorama and provides the same functionality. We build LittlePanorama on top of the classes and types defined in Panorama. We reimplement most modules with much simpler data structures, yet retaining the same program logic and function signatures. We first demonstrate that LittlePanorama is functioning properly by evaluating it against two artificial crash failures. We then compare LittlePanorama against Panorama using two real-world gray failures that are also used by the authors. Additionally, we prove that LittlePanorama is able to detect and reflect status changes of a component. Finally, we release the two bugs we acquire from the authors and publish all the scripts/software artifacts we build for other researchers who might also be interested in Panorama.

\section{Acknowledgement}
I would like to thank Peng (Ryan) Huang for being responsive. I also thank my advisors Stefan and Geoff for being supportive and understanding, otherwise I would not have been able to finished this project.