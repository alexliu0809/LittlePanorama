\begin{abstract}
Detecting failures in real world distributed systems is known to be hard. Such systems could fail for various reasons. Despite a rich literature in failure detection, many types of failures, especially gray failure, remain undiscovered and less well understood. A recent paper, titled ``Capturing and Enhancing In Situ System Observability'', made an attempt to tackle this problem. The authors proposed a system called Panorama that improved \textit{system observability} by exploiting the interactions between system components. The advantage of increased system observability is notable: new kinds of failures, such as gray failure, could now be captured; the efficiency and efficacy of failure detection are boosted; localizing the failure has also become easier. Another recent paper following this line of work even received a best paper award at NSDI '20. 

Drawn by this novel idea and recent acknowledgment it received from the research community, we decide to reproduce the original work so that we could master the idea and fully appreciate its value. We present LittlePanorama, a system that implements core modules of Panorama and provides the same functionality. To balance our workload, we reuse classes and types defined in Panorama when possible. We evaluate the performance of LittlePanorama and Panorama against bugs used in the original paper. 
\end{abstract}
